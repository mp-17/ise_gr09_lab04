\documentclass{article}

%% Language and font encodings
\usepackage[english]{babel}
\usepackage[utf8x]{inputenc}
\usepackage[T1]{fontenc}

%% Sets page size and margins
\usepackage[a4paper,top=3cm,bottom=2cm,left=3cm,right=3cm,marginparwidth=1.75cm]{geometry}

%% Useful packages
\usepackage{amsmath}
\usepackage{amsfonts} %mathematical fields fonts
\usepackage{graphicx}
\usepackage{steinmetz} %for complex numbers notation
\usepackage{float} %images held
\usepackage[colorinlistoftodos]{todonotes}
\usepackage[colorlinks=true, allcolors=blue]{hyperref}
\usepackage{xcolor} %code ambient colours
%\usepackage[colorlinks]{hyperref} %package to change colors
\usepackage{listings} %code snippet inside text
\title{Integrazione di Sistemi Embedded\\ Laboratorio 04}
\author{Matteo Perotti 251453\\ Giuseppe Puletto\\ Luca Romani 255244\\ Giuseppe Sarda 255648} 

\begin{document}
\maketitle

\newpage

\section{Introduzione}
	Il quarto laboratorio ha come obiettivo l'esplorazione delle possibilità che il comando \textbf{make} può fornire al
	programmatore durante la fase di test di un determinato progetto. Strumento fondamentale per questa fase è il file 
	\textbf{Makefile}.

\section{Approccio alla scrittura del Makefile}
	Per la scrittura del Makefile, il quale contiene tutte le "ricette" necessarie per la completa compilazione di un progetto, 
	si è consultato il manuale GNU relativo al comando  \textit{make} reperibile alla pagina web 
	\href{https://www.gnu.org/software/make/manual/make.html}{GNU make}.\\
	Di seguito sono riportate le informazioni più rilevanti ricavate dal manuale, alcune delle quali sono state usate 
	all'interno dei Makefile prodotti.

	\subsubsection*{Phony targets}
		Usando il "token" \textbf{.PHONY} si riesce ad indicare un target che non corrisponde al nome di un vero e proprio file, 
		ma ad una ricetta eseguita soltanto se espressamente richiesto. Questa direttiva è utile in caso di omonimia tra un taget 
		e un file, i quali però non sono logicamente collegati, e che quindi, senza un'indicazione specifica, sarebbero 
		erroneamente vincolati.
		Dichiarando dunque:
		\begin{lstlisting}{bash}
		.PHONY: clean
		clean:
			#do something
		\end{lstlisting}
		il target \textit{clean} viene eseguito esclusivamente quando richiesto all'esecuzione del comando make evitando eventuali malfunzionamenti nel caso esita un file nominato "clean" nella directory del progetto.

	\subsubsection*{Versioni del Makefile}
		Il Makefile è stato sviluppato in diversi modi, utilizzando regole esplicite ed implicite e variando il compilatore
		e le opzioni di compilazione in modo statico oppure dando la possibilità di scelta all'utente.
		\paragraph*{Versione 1}
		\paragraph*{Versione 2}
		\paragraph*{Versione 3}
		\paragraph*{Versione 4}
			Il Makefile è stato fatto per sfruttare il più possibile le regole implicite. Tutti i files .c vengono tradotti
			nei corrispettivi files oggetto .o.

\section{Implementazione del test}
	Per la realizzazione del target test si è deciso di dividere la verifica del codice scritto per il precedente laboratorio 
	in 4 parti:
	\begin{itemize}
		\item Creazione di uno script Python in grado di generare comandi di disegno casuali sotto specifiche
		\item Scrittura di un codice Python, basato sugli identici algoritmi usati durante il precedente lab, che si comportasse secondo specifiche relativamente alla parte funzionale ma che lasciasse totale libertà implementativa
		\item Esecuzione dei comandi generati al punto uno da parte del codice C e dello script Python
		\item Confronto dell'output finale delle due esecuzioni
	\end{itemize}
	Si noti che i limiti di questo test stanno nel fatto che, nel caso in cui gli algoritmi di disegno fossero sbagliati in 
	partenza, non c'è modo di accorgersene. I casi limite inoltre non vengono così testati perché i comandi al punto 1 vengono 
	per definizione prodotti al fine di rispettare le specifiche.

	\subsubsection*{Generazione dei comandi}
		Lo script per la generazione dei comandi accetta dall'utente un numero di comandi totali che crea usando la funzione \textit{randint} contenuta nel modulo \textit{random}. Infine i comandi generati vengono scritti all'interno di un file che verrà poi passato ai file eseguibile.
		L'esecuzione finisce con la scrittura di un \textbf{token} che segnala la fine dello stream di comandi.

	\subsubsection*{Script di test}
		Lo script di test che produce il risultato esatto dall'esecuzione dei comandi, come già specificato in precedenza, si basa sugli stessi algoritmi di riferimento del codice C. Genera e modifica una matrice di interi più facilmente accedibile e 
		modificabile che della struttura usata nel codice originale. Ulteriori vantaggi dell'utilizzo del python sono dovuti alla completa assenza di interfacce e libertà di gestione della realizzazione del codice. 

	\subsubsection*{Modifica del programma}
		Il programma è stato modificato per poter riconoscere il comando di QUITTING ('Q'). Appena esso viene ricevuto
		viene chiamata una funzione che apre un file, disegna la matrice (frameBuffer) e poi lo chiude. In questo modo 
		diventa semplice il confronto tra le due matrici generate dai programmi scritti in diversi linguaggi.
		L'interfaccia del programma è stata anche regolarizzata per rientrare nelle specifiche di interfaccia.

\end{document}
